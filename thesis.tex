
\documentclass[onehalf,11pt]{beavtex}
\title{GPU-Based Fluid Structure Interaction using Immersed Boundary Methods}
\author{Christopher Minar}
\degree{Masters of Science}
\doctype{Masters Thesis}
\department{Mechanical Industrial and Manufacturing Engineering}
\depttype{School}
\depthead{Director}
\major{Mechanical Engineering}
\advisor{Kyle Niemeyer}
\submitdate{September 23, 2016}
\commencementyear{2016}
\abstract{Engineering applications often require fast, accurate solutions of fluid flow around freely moving bodies.
The massive parallelism enabled by GPU architecture enables high performance, offering a promising alternative to traditional solver acceleration via multicore CPUs.
However, fully harnessing GPU parallelism requires specialized algorithms and computing strategies.
This work modifies a direct-forcing immersed boundary method to model fluid-structure interaction and investigates its behavior on GPUs.
We performed verification of our solver using lid-driven cavity flow and impulsively started flow over a cylinder, and investigated its behavior with flow over a forced oscillating cylinder.}
\acknowledgements{We gratefully acknowledge the support of NVIDIA Corporation with the donation of the Tesla K\subsubsection0 GPU used for this research.
We also thank the Barba Group for developing, maintaining, and distributing their cuIBM code.}

\usepackage{algorithm}
\usepackage{algorithmic}
\usepackage[textsize=footnotesize]{todonotes}%used for inline todo notes with \todo[]{}
\usepackage{latexsym,amsmath,amssymb} %math packages
\usepackage{tikz} %used for tikz graphics
\usepackage{standalone}  %used for tikz graphics
\usetikzlibrary{shapes}  %used for tikz graphics
\usepackage{graphicx} %needed for subfigure
\usepackage{float,caption,subcaption} %figure stuff
\graphicspath{{./figure/}} %figure path
\usepackage{siunitx}
\sisetup{group-separator={,},
	detect-all,
	binary-units,
	list-units = single,
	range-units = single,
	tophrase = --,
	per-mode = symbol-or-fraction,
	separate-uncertainty = true,
	list-final-separator = {, and }
}


\begin{document}
\todo[inline]{fix margins}
\todo[inline]{change Fadlun to fadlun et al}
\maketitle

\mainmatter

\chapter{Introduction}
Immersed boundary methods (IBMs) are a broad term referring to of group of approaches used for simulating fluid flow over complex bodies.
The goal of IBMs is to represent a surface on a structured grid, removing the need to create a body fitted mesh.
These techniques are well-suited for simulating flow involving complex moving bodies because they do not require re-meshing between time steps.
We have developed an IBM solver for operation on graphics processing units (GPUs) designed to handle incompressible fluid--structure interaction problems with rigid bodies.

Peskin~\cite{Peskin:1972gh} first introduced the IBM in the 1970s to model blood flow with elastic boundaries.
The original IBM adds a forcing term to the Navier--Stokes equations that only acts near a body.
This forcing term simulates the force the body applies on the fluid, and is modeled as a spring.
Peskin's IBM was designed to handle elastic, deforming bodies.
One way to effectively implement Peskin's IBM for rigid bodies is to set a large spring coefficient; however, this makes the equations numerically stiff.

The IBM has since evolved to be applicable to a wide range of problems, as reviewed by Mittal and Iaccarino~\cite{Mittal:2005ii} and more recently Sotiropoulos and Yang~\cite{Sotiropoulos:2014gv}.
Mohd-Yusof~\cite{MohdYusof:1997wh} and Fadlun et al.~\cite{Fadlun:2000fl} developed a version for rigid bodies called the direct forcing method.
Instead of solving for the added force term in the Navier--Stokes equation, the velocities at the nodes nearest to the body are interpolated for using the body velocity to enforce the no-slip condition.
However, using the direct forcing method with a moving body causes numerical oscillations~\cite{liao2010simulating}~\cite{Luo:2012gx}.
This effect is manageable for preset motion but with a freely moving body the solver will poorly predict body position, velocity, and forces.
The numerical oscillations are caused by nodes switching representations between time steps, e.g., a node near the body (and thus treated differently) becoming a node solved for with the Navier--Stokes equations.
The approach of Luo et al.~\cite{Luo:2012gx} deals with the numerical oscillations by using a weighting function to transition between the two schemes.

The goal of this project is to develop a GPU-based solver capable of simulating 2D, incompressible flow over a moving, complex body. 
\todo[inline]{write end purpose of solver}
\todo[inline]{give solver a name}
\todo[inline]{add segment about lee and lee}
Chapter \ref{Numerical Methods} will discuss the theory of several direct forcing methods and their applicability to freely moving bodies. 
Chapter \ref{Implementation} cover the discretization of the methods as well as the challenges, strategies and implementation of the GPU.
Chapter \ref{Validation} will show the validation and verification of different aspects of the solver using five different test cases; lid driven cavity, flow over an impulsively started cylinder, forced motion of a cylinder in static flow, forced motion of a cylinder in flow and vorticity induced vibrations (VIV).
\todo[inline]{make sure that is the correct words for VIV}
The final chapter will discuss the results, go over problems and recommend future work.

\chapter{Numerical Methods}\label{Numerical Methods}
Fluid flow is governed by the two-dimensional, incompressible form of the Navier--Stokes and mass continuity equations:
\begin{align}
\frac{\partial \textbf{u}}{\partial t} + \nabla ( \textbf{uu} ) &= -\nabla p + \nu\nabla^{2}\textbf{u} + \textbf{f} \label{eq:NavierStokes} \\
\nabla \cdot \textbf{u} &= 0 \label{eq:Continuity} \;.
\end{align}
Here, $u$ is velocity, $p$ is pressure and $\nu$ is the constant kinematic viscosity.
\todo[inline]{what is f}

\section{Navier--Stokes Fractional Step}\label{NM:NavierStokes}
Equations \eqref{eq:NavierStokes} and \eqref{eq:Continuity} are solved using the fractional step method (e.g., \cite{Perot1993}). 
In the first step, an advection-diffusion equation without pressure is used to calculate an intermediate velocity:
\begin{equation}\label{eq:Intermediate Velocity}
\hat{\textbf{u}} - \frac{\Delta t}{2}L(\hat{\textbf{u}}) = \textbf{u}^n + \Delta t(\textbf{RHS}^n) \;,
\end{equation}
where $\hat{\textbf{u}}$ is the intermediate velocity, $\Delta t$ is the time step, $L$ is the Laplacian operator and $\textbf{RHS}$ is the explicit, discretized advection and diffusion terms.
The superscript $n$ represents the time step.
In time, second-order Adams--Bashforth and Crank--Nicolson methods are used to discretize the advection and diffustion terms, respectively.
\todo[inline]{move this line to the implementation section}

In the second step, the continuity equation is imposed to approximate the pressure, $p$:
\begin{equation}\label{eq:Poisson}
\nabla^2p^{n+1} = - \frac{\nabla\hat{\textbf{u}}}{\Delta t} \;,
\end{equation}

Velocity is updated in the last step with:
\begin{equation}\label{eq:Projection}
\textbf{u}^{n+1} = \hat{\textbf{u}} - \Delta t\nabla p^{n+1} \;.
\end{equation}

\section{Node Identification}
To facilitate the discussion of IBMs the nomenclature with regards to point identification will be touched on here. 
The nomenclature is mostly adopted from the work of Luo et al.\cite{Luo:2012gx}
\todo[inline]{add figure for node identification}
\todo[inline]{add text for node identification}

\begin{figure}[htb] %this should probably go somehwer eelse
    \centering
    \includestandalone[width=0.5\linewidth]{basic_interp_figure}
    \caption{A simplified diagram of linear interpolation for velocity values at the hybrid node}
    \label{fig:2}
\end{figure}

\section{Fadlun}
In Peskin's\cite{Peskin:1972gh} IBM the forcing term $f$ in Equation \eqref{eq:NavierStokes} is solved for at the hybrid nodes in a way that enforces no-slip. 
In the direct forcing methods developed by Fadlun\cite{Fadlun:2000fl} and Yusof\cite{MohdYusof:1997wh} the navier stokes equation isn't solved at the hybrid nodes. 
Instead, the velocity at the hybrid node is approximated by linearly interpolating between the second hybrid node and the body surface. 
As explained by Fadlun it is appropriate to enforce this approximation in the intermediate velocity step.
\todo[inline]{why is it appropriate to enforce this in the intermediate velocity step}
\todo[inline]{when is this approximation not viable?}
\todo[inline]{figure to demonstrate linear interpolation}
$\textbf{RHS}$ and $1-\frac{\Delta t}{2}L$ from Equation \eqref{eq:Intermediate Velocity} can be setup as if they was no body, then modified at the hybrid nodes to enforce no slip with the linear interpolation.
The discretization of this is shown in appendix \ref{Fadlun Linear Interpolation}
\todo[inline]{add reference to the discretization of the normal part of RHS and LHS}
The Poisson Equation \ref{eq:Poisson} and projection step \ref{eq:Projection} are left unmodified in the work of Fadlun et al.

\section{Modified Fadlun}
If the Poisson equation  equation to account for the presence of a body, the continuity equation will not be satisfied. 
Depending on the type of simulation this can create significant creation or destruction of mass at the body, leading to poor solver accuracy.
Researchers following \todo[]{find researchers} Fadlun et al. suggested several different ways to solve this problem.
\todo[inline]{outline lee lee method}
\todo[inline]{explain my method}

\section{Luo}

Direct forcing methods suffer from numerical oscillation when the immersed body is moving.
\todo[inline]{cite oscillations in direct forcing}
As the body moves, background nodes change how they calculate values. 
For example Figure~\ref{fig:Temporal Discontinuity} depicts a bulk fluid $u$ velocity node at time step $t^n$ become a hybrid node at time step $t^{n+1}$.
As the node transitions the intermediate velocity will change from being calculated by Equation ~\eqref{eq:Intermediate Velocity} to Equation.
\todo[inline]{add equation and reference to equation the does the linear interpolation}
Equations \eqref{eq:Intermediate Velocity} and \todo[]{add eq ref} are both correct representations for the intermediate velocity but they have different errors associated with them which causes the calculated velocity to be slightly different.
This discontinuous change in velocity is responsible for the numerical oscillations.
\todo[inline]{reference luo et al somewhere in this explanation}
Note that this effect happens whenever any node transitions, not just the intermediate $u$ velocity nodes.
\todo[inline]{give a more in depth look at the math that happens which causes this error to happen?}%maybe this should go somewhere else.

\begin{figure}[htb]
    \centering
    \includestandalone[width=0.5\linewidth]{figure/node_transition_figure}
    \caption{A diagram illustrating the immersed body's interaction with the background grid that causes numerical oscillations.}
    \label{fig:Temporal Discontinuity}
\end{figure}

The field values (velocity and pressure) are extrapolated across the body such that Eq.~\eqref{eq:NavierStokes} can be solved at the hybrid nodes using a standard stencil. 
This process is elaborated on in section \ref{Sec:Field Extrapolation}. 
In the method proposed by Luo et al.~\cite{Luo:2012gx}, the field values at the hybrid nodes are calculated via both Navier--Stokes and interpolation. 
The final field values are calculated using a weighted combination of both solutions.
The weighting is designed smooth the transition between solutions. 
When the hybrid node is close to the body, it will be dominated by the interpolated solution, while a the hybrid node far from the body will be dominated by the Navier--Stokes solution. 
This process is elaborated on in section \ref{Sec:Weighting}

\subsection{Field extrapolation to ghost nodes}
\label{Sec:Field Extrapolation}

Navier--Stokes can be used to calculate values at the hybrid nodes if the ghost nodes are modified to correctly represent the presence of the body.
A simple example of this can be seen in Figure~\ref{Fig: Simple Interpolation}.
If the slope of velocity between the hybrid node and the body is assumed to be linear then the slope can be extrapolated inwards to the ghost node.
When Navier--Stokes is then solved, the fluid velocity at the body surface will be calculated as the velocity of the body, mimicking the no slip boundary condition.
\begin{figure}[htb]
	\centering
	\includestandalone[width=0.5\linewidth]{basic_extrapolation}
	\caption{A simple 1D, linear extrapolation example}
	\label{Fig: Simple Interpolation}
\end{figure}

Field values are extrapolated across the body using a 2D bilinear method described in Luo et al.~\cite{Luo:2012gx}.
Unlike Luo et al., the solver uses a staggered grid, causing the extrapolation process to have three variations; one for pressure nodes and ones for u and v velocity nodes.
Field values are first interpolated at an image point (IP) outside the body using the surrounding field values and the boundary condition. 
The 2D bilinear interpolation approximates the field between four nodes surrounding the IP with Equation~\eqref{eq:Interpolation}.
\todo[inline]{move equation up}
The four interpolation nodes are used to set up a system of equations to solve for the coefficients, which can then be used to solve for the field values. 
Some of the interpolation nodes will be moved to account for the presence of the body as seen in Figure~\ref{fig:Ghost node extrapolation}.
\begin{figure}[htb]
	\centering
	\includestandalone[width=0.5\linewidth]{GN_interp_figure}
	\caption{A detailed schematic u velocity extrapolation to ghost nodes.}
	\label{fig:Ghost node extrapolation}
\end{figure}

The ghost nodes(GN), indicated by solid squares, are projected onto the surface to find the body intercept(BI), indicated by the thick open circle.
\todo[inline]{change diagram to not have two open circles, its confusing}
Body nodes used to track the body's position are not the same as the BI and the two are typically not coincident.
If the body has curvature, the body intercept will not fall exactly on the body.
The line drawn between the BI and GN will be perpendicular to the tangent line at the BI.
That line is then mirrored over the surface to find the IP, indicated by the solid triangle. 

The four field values surrounding the IP are used to interpolate for the field value at the IP using equation~\eqref{eq:Interpolation}. 
If one of the four interpolation nodes is inside the body, as seen in the extrapolation for GN 1 in Figure~\ref{fig:Ghost node extrapolation}, then that node is moved to the BI and the boundary condition is used.
If none of the four field values are inside the body as seen in the extrapolation for GN 2 then the node closest to the BI is moved to the BI.
To satisfy the no slip condition at the body, a Dirichlet boundary condition equal to the body velocity is used when extrapolating for velocity GNs. 
The Neumann Equation~\eqref{eq:Neumann}, is used to approximate the pressure boundary condition.
A system of equations is set up to solve for the $a$ coefficients in Equation~\eqref{eq:Interpolation}.
Pressure nodes on the body use Equation~\eqref{eq:Neumann Node} in the system, which is simply the derivative of Equation~\eqref{eq:Interpolation}. 
\begin{align}
\phi (x,y) = a_0 + a_1 x + a_2y + a_3 x y \label{eq:Interpolation} \\
\phi (x,y) = a_1 + a_2 + a_3 (x+y) \label{eq:Neumann Node} \\
\left. \frac{\partial p}{\partial \textbf{n}}\right|_{BI} = \left. -\rho \frac{D\textbf{u}}{Dt}\cdot \textbf{n}\right|_{BI}
\label{eq:Neumann}\;.
\end{align}
$\phi$ is the variable being interpolated ($u$, $v$, or pressure); $a_0$, $a_1$, $a_2$ and $a_3$ are coefficients, and $x$ and $y$ give the node location.
$\rho$ is the density, $D\textbf{u}\left/Dt\right.$ is the material derivative, and $\textbf{n}$ is the unit vector normal to the body.
Solving this system on the GPU requires direct inversions of the $4 \times 4$ matrices.
This process is elaborated on in \ref{system of euqations}.
Once the $a$ coefficients have been found, the field value at the IP can be calculated with equation~\eqref{eq:Interpolation} and extrapolated across the surface using equation~\eqref{eq:Velocity Interpolation} for velocity or ~\eqref{eq:Pressure Interpolation} for pressure. 
$\Delta l$ is the distance from GN to IP.
\begin{align}
\textbf{u}_{GN} &= 2\textbf{u}_{BI} - \textbf{u}_{IP} \label{eq:Velocity Interpolation} \\
p_{GN} &= p_{ip} - \Delta l \left. \frac{\partial p}{\partial \textbf{n}}\right|_{BI} \;, \label{eq:Pressure Interpolation}
\end{align}

\subsection{Interpolation to hybrid nodes}
\label{Sec:Interpolation}

Interpolating for the field value at the hybrid node is largely the same as interpolating for the ghost nodes' image point described previously.
The BI is once again found by projecting the hybrid node along the surface normal.
The IP is the same distance from the BI to the HN in the opposite direction of the HN as seen in Figure~\ref{fig:Interpolate}.
The IP is only used to find the four interpolation nodes, one of which will always be the HN.
The interpolation node coincident with the HN is moved to the BI and the appropriate boundary condition for velocity or pressure is applied.
Once equation~\eqref{eq:Interpolation} is solved, the hybrid node is interpolated for.
Details on this process can be found in \ref{a: interpolation to hybrid nodes}.
\begin{figure}[htb]
	\centering
	\includestandalone[width=0.5\linewidth]{HN_interp_figure}
	\caption{A detailed schematic of the hybrid node interpolation.}
	\label{fig:Interpolate}
\end{figure}

\subsection{Weighting}
\label{Sec:Weighting}

To dampen numerical oscillations, Luo et al.~\cite{Luo:2012gx} introduced a smooth transition between solution regimes at the body instead of a Dirac-delta-type function of traditional direct forcing.
The scalar $\alpha$ is introduced to the solutions for intermediate velocity and the pressure:
\begin{equation}\label{eq:Weight}
\theta = \left(1-\alpha \right)\theta_{Navier-Stokes} + \alpha \theta_{Interpolated} \;,
\end{equation}
where $\theta$ represents either intermediate velocity or pressure.
Transitioning between the Navier--Stokes and interpolated solutions should meet three criteria:
\begin{enumerate}
	\item Hybrid nodes farther from the body should favor the Navier-Stokes solution ($\alpha \rightarrow 0$ as distance $\uparrow$).
	\item Hybrid nodes closer to the body should favor the interpolated solution ($\alpha \rightarrow 1$ as distance $ \rightarrow 0$).
	\item The weighting function should be smooth and continuous as the hybrid node moves away from the body.
\end{enumerate}
To solve for $\alpha$, it is assumed that each hybrid node has at most two neighboring ghost nodes (this will not be true for sharp corners):
\begin{equation}
\alpha = \sqrt{\left(\frac{\Delta_1}{\Delta x}\right)^2 + \left(\frac{\Delta_2}{\Delta y}\right)^2} \;.
\label{eq:Alpha}
\end{equation}
where $\Delta_1$ and $\Delta_2$ correlate to GN$_1$ and GN$_2$, respectively, and $\Delta x$ and $\Delta y$ are the grid spacing in the $x$ and $y$ directions, respectively.
As described in Luo et al. and shown in Figure~\ref{fig:Weight}, $\Delta_1$ and $\Delta_2$ are the closest distances between the body and the $x$ and $y$ ghost nodes, respectively.
If the hybrid node only has one neighbor, then $\Delta=0$ for the other direction.
\begin{figure}[htb]
	\centering
	\includestandalone[width=0.5\textwidth]{alpha_figure}
	\caption{Schematic of the calculation of $\alpha$ for u velocity nodes.}
	\label{fig:Weight}
\end{figure}

\subsection{Putting it all together}
\label{putting it all together}

Luo et al. did not describe exactly where all this interpolating and extrapolating takes place.
It is a bit confusing where and when values should be extrapolated and interpolated.
We tried doing it two ways.
The first way more closely follows the work of Luo et al.\cite{Luo:2012gx} which says to perform the interpolations, extrapolations and weighting inside of the linear algebra solutions for the intermediate velocity and pressure. 
Field values will solved by plugging in Equations~\eqref{eq:Intermediate Velocity} or \eqref{eq:Poisson} and \eqref{eq:Interpolation} into Equation~\eqref{eq:Weight}. 
The discretization of these can be seen in section \ref{ID:Iterative}.
We also tried to resolve the time step in a non-iterative manner by calculating the interpolations, extrapolations and weighting outside of the linear algebra. 
The order is setup such that a value is interpolated or extrapolated just before it is needed for the subsequent step.
We came up with the following sequence that is elaborated on in section \ref{ID:Serial}:
\begin{enumerate}
	\item Identify all nodes as hybrid, ghost, fluid or solid.
	\item Extrapolate u and v velocity values to ghost nodes.
	\item Interpolate for u and v velocities at hybrid nodes.
	\item Calculate right hand side for the intermediate velocity solution.
	\item Solve for intermediate velocity.
	\item Weight the two intermediate velocity solutions.
	\item Extrapolate intermediate velocities to ghost nodes.
	\item Calculate right hand side for the Poisson equation.
	\item Solve the Poisson equation.
	\item Interpolate for pressure at hybrid nodes.
	\item Weight the two pressure solutions.
	\item Extrapolate pressure to ghost nodes.
	\item Update velocity.
	\item Advance time.
\end{enumerate}

\chapter{Implementaiton and Discretization}\label{Implementation}

This chapter will detail the discretization of the immersed boundary method and implementation on the GPU. 

\section{GPU implementation strategy}
\label{GPU implementation strategy}
\todo[inline]{store everything on the gpu}
\todo[inline]{all calculations on the gpu}
\todo[inline]{no transferring data}
\todo[inline]{don't optimize things other than poisson eq}
\todo[inline]{this hasn't been edited from the paper yet}

We developed our solver for full GPU operation to improve performance relative to CPU-based algorithms.
To this end, several overarching strategies have been implemented.
First, all calculations are performed on the GPU, and all data is stored on the GPU.
Typically, efforts are taken to avoid thread divergence when writing code for a GPU.
We found it favorable to write kernels with occasionally poor thread-parallel performance, i.e., kernels with significant divergence, rather than copying data to the host to do calculations to reduce divergence.
In our experiences, transferring data back and forth took significantly more time than would be saved by running some operations on a CPU.
This was particularly relevant when moving from a stationary body to a moving body.
When simulating a stationary body, several calculations only need to be performed once, such as the left side matrices or node identification.
When simulating a moving body, these values need to be recalculated every time step, causing a total computational cost for each nearly as high as the Poisson equation solve (typically the most expensive step).
This relates to the second strategy: evaluate algorithm performance compared to the Poisson equation.
Increasing the parallel performance of kernels has little effect on the overall solver performance because solving the Poisson equation typically takes more than \SI{90}{\percent} of the total computational time.

The third strategy comes from Layton et al.~\cite{layton2011cuibm}.
When using the Cusp library to perform the multigrid method, creating the preconditioner can take as much time as solving the Poisson equation.
In a moving-body simulation, the preconditioner is normally recalculated each time step, but Layton et al.~\cite{layton2011cuibm} found it possible to skip some time steps without loss of fidelity.
Here, we recalculate the preconditioner only if solution of the previous time step required more than 100 iterations.

\section{Grid Staggering}
\label{Grid Staggering}

A collocated grid stores all variable information at the center of each cell.
When discretized on a collocated grid, Navier--Stokes will yield an odd-even decoupling of pressure an velocity resulting in a checkerboard pattern. 
\todo[inline]{findsource for this info}
Using a staggered grid is a relatively straightforward way to avoid odd-even decoupling.
A staggered grid stores scalar values(pressure, temperature, density etc.) at the cell center and velocities at the cell face. 
Staggered grids also do not require a pressure boundary condition.
\todo[inline]{is this true?}
Figure~\ref{fig:stagger} shows how the grid is staggered.
It is worth noting that Luo et al.\cite{Luo:2012gx} uses a collocated grid.
Due to the sensitivity of immersed boundary methods near the body, this will cause some difference.
There is no theoretical reason, however, why the numerical suppression method will not work on a staggered grid.
\begin{figure}[htb]
	\centering
	\includestandalone[width=0.5\textwidth]{grid_stagger}
	\caption{Grid staggering: The velocities are staggered in the positive x and y directions.}
	\label{fig:stagger}
\end{figure}

\section{Node Identification}
\todo[inline]{find krishnan paper to reference}
\todo[inline]{find algorithm to reference}

\section{Force Calculation}
\label{Force Calculation}
\todo[inline]{find reference to control volume approach}
\todo[inline]{write control volume approach}
\todo[inline]{write luo et al approach}
\todo[inline]{find reference to original luo approach}

\section{Preconditioning and Linear Algebra Solvers}
\label{Preconditioning and Linear Algebra Solvers}
\todo[inline]{talk about cusp}
\todo[inline]{talk about the preconditioners}
\todo[inline]{talk about multi grid methods}
\section{Navier--Stokes}
\label{ID:Navier Stokes}

The 2D incompressible Navier--Stokes equation forms the backbone of the immersed boundary method.
To solve Navier--Stokes, the fractional step method is used.
In the first step, Equation~\eqref{eq:Incompressible NavierStokes} drops the pressure term and is discretized to get an intermediate velocity.

\subsection{Intermediate Velocity}
\begin{equation}
\frac{\partial \textbf{u}}{\partial t} + \nabla ( \textbf{uu} ) = -\nabla p + \nu\nabla^{2}\textbf{u} \label{eq:Incompressible NavierStokes}
\end{equation}
The advection term, $\nabla (\textbf{uu})$, is discretized in time using explicit, second-order Adams-Bashforth.
Diffusion, $\nu \nabla^2 \textbf{u}$, is discretized using Crank-Nicolson.
The result is an expanded form of Equation\eqref{eq:Intermediate Velocity}, Equation~\eqref{eq:Expanded Intermediate Velocity}.
\begin{equation}
\label{eq:Expanded Intermediate Velocity}
\hat{\textbf{u}} - \frac{\Delta t}{2}L(\hat{\textbf{u}}) = \textbf{u}^n + \Delta t\left(0.5L(\textbf{u}^n) - 1.5N(\textbf{u}^n) + 0.5N(\textbf{u}^{n-1}) + BC\right)
\end{equation}
Where $N$ is the advection operator, $L$ is the diffusion Laplacian operator and BC is the boundary condition term.
\subsubsection{Advection}\label{ID: Advection}
The advection term, expands to become Equations~\eqref{eq: Advection 1} and~\eqref{eq: Advection 2}. 
Because the grid is staggered, the discretization of Equations~\eqref{eq: Advection 1} and~\eqref{eq: Advection 2} are slightly different.
For simplicity only the discretization of the u equation will be shown.
Note that $u$ and $v$ represent scalar velocity and $\textbf{u}$ is vector velocity.
\begin{align}
&u\frac{\partial u}{\partial x} + v\frac{\partial u}{\partial y} \label{eq: Advection 1} \\ 
&u\frac{\partial v}{\partial x} + v\frac{\partial v}{\partial y} \; \label{eq: Advection 2}
\end{align}
Discretizing on a uniform grid using second order central differencing results in Equation \eqref{eq: discretized uniform advection}.
Figure~\ref{fig:discretized uniform advection} shows a uniform grid.
\begin{figure}[htb]
	\centering
	\includestandalone[width=0.5\textwidth]{advection_discretization}
	\caption{A uniform grid reference to be used with Equation~\eqref{eq: discretized uniform advection}}
	\label{fig:discretized uniform advection}
\end{figure}
\begin{equation}
u_{i,j}\frac{u_{i+1,j} - u_{i-1,j}}{2dx} + \frac{v_{i,j} + v_{i+1,j} + v_{i,j-1} + v_{i+1,j-1}}{4}\frac{u_{i,j+1} - u_{i,j-1}}{2dy} \label{eq: discretized uniform advection}
\end{equation}

A non-uniform grid requires a little more involved discretization. 
If $\Delta x_{i+1}$ does not equal $\Delta x$ then $\frac{\partial u}{\partial x}$ must be calculated at the cell centers, $p_{i,j}$ and $p_{i+1,j}$, then interpolated to $u_{i,j}$. 
\todo[inline]{add figure and equation for dudx interp}
In addition, $v$ will be bilinearly interpolated for at $u_{i,j}$.
Figure~\ref{fig:bi-linear-interpolation} shows a schematic of bilinear interpolation.
First, $v$ at 1 and 2 are linearly interpolated for using Equations \eqref{eq:bili 1} and \eqref{eq:bili 2}.
Then $v_1$ and $v_2$ are used to linearly interpolated for $v$ at $u_{i,j}$ using Equation \eqref{eq:bili 3}.
\begin{align}
v_1 &\approx \frac{x_2 - x}{x_2 - x_1}v_{i,j} + \frac{x - x_1}{x_2 - x_1}v_{i+1,j} \label{eq:bili 1} \\
v_2 &\approx \frac{x_2 - x}{x_2 - x_1}v_{i,j-1} + \frac{x - x_1}{x_2 - x_1}v_{i+1,j-1} \label{eq:bili 2} \\
v_x &\approx \frac{y_2 - y}{y_2 - y_1}v_{1} + \frac{y - y_1}{y_2 - y_1}v_{2} \; \label{eq:bili 3}
\end{align}
\begin{figure}[htb]
	\centering
	\includestandalone[width=0.5\textwidth]{bi-linear_interpolation}
	\caption{Non-uniform bi linear interpolation.}
	\label{fig:bi-linear-interpolation}
\end{figure}

\subsubsection{Diffusion}
The diffusion term, $\nu \nabla^2 \textbf{u}$, expands to become Equations \eqref{eq:ID Diffusion x} and \eqref{eq:ID Diffusion y} for the u and v momentum equations, respectively.
\begin{align}
& \frac{1}{Re}\left(\frac{\partial^2 u}{\partial x^2} + \frac{\partial^2 u}{\partial y^2}\right)\label{eq:ID Diffusion x} \\
& \frac{1}{Re}\left(\frac{\partial^2 v}{\partial x^2} + \frac{\partial^2 v}{\partial y^2}\right)\; \label{eq:ID Diffusion y}
\end{align}
$\nu$ can be simplified to 1 over the Reynolds number so long as the characteristic velocity is 1. 
As with advection, the discretization for \eqref{eq:ID Diffusion x} and \eqref{eq:ID Diffusion y} are different because they are computed at the u and v nodes.
The discretization is also more complex when done on a non-uniform grid.
Equation \eqref{eq:ID Diffusion 1} is \eqref{eq:ID Diffusion x} discretized over a uniform grid.
\begin{equation}
\frac{u_{i+1,j} + u_{i-1,j} + u_{i,j+1} + u_{i,j-1} - 4u_{i,j}}{\Delta^2}\label{eq:ID Diffusion 1}
\end{equation} 
To discretize over an non-uniform grid like that in Figure \ref{fig:ID diffusion} the terms in Equation \eqref{eq:ID Diffusion x} are expanded.
For example $\frac{\partial^2 u}{\partial x^2}$ will become $\frac{\partial}{\partial x}\left(\frac{\partial u}{\partial x}\right)$.
The partial derivative of $u$ with respect to $x$ is calculated at the cell centers adjacent to $u_{i,j}$ and the partial derivative of that is taken with respect to $x$ at $u_{i,j}.$
For simplicity, only the discretization of $\frac{\partial^2 u}{\partial x^2}$ is show in Equation \eqref{eq:ID Diffusion 2} for a non-uniform grid because the discretization of $\frac{\partial^2 u}{\partial y^2}$ follows the same pattern.
\begin{equation}
\frac{u_{i+1,j} - u_{i,j}}{dx_{i+1}(dx_{i+1}+dx_i)0.5} + \frac{u_{i-1,j} - u_{i,j}}{dx_{i}(dx_{i+1}+dx_i)0.5}
\label{eq:ID Diffusion 2}
\end{equation}
\begin{figure}[htb]
	\centering
	\includestandalone[width=0.5\textwidth]{Diffusion_discretization}
	\caption{Non-uniform discretization example of the diffusion term.}
	\label{fig:ID diffusion}
\end{figure}

\subsubsection{Boundary Terms}
\begin{figure}[htb]
	\centering
	\includestandalone[width=0.5\textwidth]{bc_discretization}
	\caption{North-East Boundary Condition Discretization.}
\end{figure}
\todo[inline]{show how being at the boundary effects diffusion and advection}
\subsubsection{Left Side Matrix}
\subsection{Poisson}
\subsubsection{Right Side Matrix}
\subsubsection{BC}
\subsubsection{Left Side Matrix}
\section{Fadlun Immersed Boundary Method}
\subsection{other body stuff?}
\subsection{Intermediate Velocity}
\subsubsection{Interpolation}
\subsubsection{Left Side Matrix} 
\section{Luo Immersed Boundary Method}
\section{Serial Solution}\label{ID:Serial}
\subsection{Field extrapolation to ghost nodes}
\subsection{Interpolation to hybrid nodes}
\subsection{Weighting}
\subsection{Full time step}
\section{Iterative Solution}\label{ID:Iterative}
\subsection{Intermediate Velocity}\label{ID:Iterative Intermediate Velocity}
\subsection{Poisson}\label{ID:Iterative Poisson}


\chapter{Validation and Verification}\label{Validation}
\section{Lid Driven Cavity}
\section{Impulsively started cylinder}
\subsection{luo}
\subsection{luo iter}
\subsection{fadlun modified}
\subsection{fadlun?}
\section{Static cylinder}
\subsection{fadlun modified}
\subsection{luo}
\subsection{luo iter}
\section{osc cylinder}
\subsection{fadlunmodified}
\subsection{luo}
\subsection{luo iter}
\section{viv}
\subsection{fadlun modified}
\subsection{luo}
\subsection{luo iter}
\chapter{conclusions}
\chapter{future work}


%We're born meat and we die meat. Meanwhile, we learn (see Algorithm \ref{alg:learning}).

%\begin{algorithm}[h]
%\caption{\textsc{Learning}}
%\label{alg:learning}
%\begin{algorithmic}[\chapter]
%\ENSURE{Optimal policy $\mathcal{C}$}
%	\STATE $\mathcal{C} \gets 32$
%	\RETURN $\mathcal{C}$
%\end{algorithmic}
%\end{algorithm}

%Wow, that really was excellent.
%This is the end, my only friend, the end.

%\begin{figure}[!ht]
%\centering
%\fbox{\huge Box}
%\caption{Go figure.}
%\end{figure}



\bibliographystyle{plain}
\bibliography{thesis}

\appendix
\chapter{Discretization}
\section{Fadlun}
\subsection{Fadlun Linear Interpolation}\label{Fadlun Linear Interpolation}
\section{Luo}
\subsection{Solving a system of four eqations on the gpu}\label{system of euqations}
\subsection{a: interpolation to hybrid nodes}
\label{a: interpolation to hybrid nodes}

\end{document}
